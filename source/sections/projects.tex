\section{Projects}

\cventry
{Nest.js module for building Slack applications.}
{nestjs-slack-listener}
{\href{https://github.com/hanchchch/nestjs-slack-listener}{Github}}
{03/2022}
\begin{cvitems}
\item {Wrote a blog post about this project \href{https://velog.io/@hanchchch/%EC%A7%84%EC%8B%AC%EC%9C%BC%EB%A1%9C-%EC%97%85%EB%AC%B4-%EC%9E%90%EB%8F%99%ED%99%94-%EC%8A%AC%EB%9E%99%EB%B4%87-%EB%A7%8C%EB%93%A4%EA%B8%B0-1}{(link)}}
\item {Developed decorators that help build Slack applications with Nest.js, in typical Nest.js-style}
\item {Published on npm}
\end{cvitems}

\cventry
{Automated Minecraft server cloud provisioning tool}
{i-wanna-play-minecraft}
{\href{https://github.com/hanchchch/i-wanna-play-minecraft}{Github}}
{10/2022}
\begin{cvitems}
\item {For those who want to serve Minecraft server but poor}
\item {Provision a running Minecraft server instance in cloud instantly with only one command}
\item {Pause, restart, or even destroy the instance with one command}
\item {Specify amount of time to run the server to reduce the costs}
\item {Used Pulumi and Go client of it}
\end{cvitems}

\cventry
{Cloud-based malicious URL detection system.}
{gimi}
{\href{https://github.com/hanchchch/gimi}{Github}}
{09/2022 - On going}
\begin{cvitems}
\item {Inspect the URL using containers and detect whether it is malicious or not}
\item {Provision a new container for each URL to inspect it in isolation, just like FaaS}
\item {Built a simple FaaS platform, to operate heterogenous functions in cloud}
\item {Used NestJS for controllers and micro-services, Docker Go client for container management, protobuf for communications}
\end{cvitems}
