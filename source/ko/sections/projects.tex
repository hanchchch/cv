\section{Projects}

\cventry
{Nest.js module for building Slack applications.}
{nestjs-slack-listener}
{\href{https://github.com/hanchchch/nestjs-slack-listener}{Github}}
{03/2022}
\begin{cvitems}
\item {블로그에 이 프로젝트와 관련된 포스팅을 게시했습니다. \href{https://velog.io/@hanchchch/%EC%A7%84%EC%8B%AC%EC%9C%BC%EB%A1%9C-%EC%97%85%EB%AC%B4-%EC%9E%90%EB%8F%99%ED%99%94-%EC%8A%AC%EB%9E%99%EB%B4%87-%EB%A7%8C%EB%93%A4%EA%B8%B0-1}{(링크)}}
\item {NestJS용 모듈 및 데코레이터를 개발해, Slack 앱을 일반적인 NestJS 스타일로 쉽게 제작할 수 있도록 했습니다.}
\item {npm에 배포했습니다. \href{https://www.npmjs.com/package/nestjs-slack-listener}{(링크)}}
\end{cvitems}

\cventry
{Automated Minecraft server cloud provisioning tool}
{i-wanna-play-minecraft}
{\href{https://github.com/hanchchch/i-wanna-play-minecraft}{Github}}
{10/2022}
\begin{cvitems}
\item {Minecraft 서버를 배포하고 싶지만 그럴만한 예산이 부족한 사람들을 위해 개발했습니다.}
\item {하나의 명령어으로 Minecraft 서버를 클라우드에 자동으로 프로비저닝합니다.}
\item {일시정지, 재시작, 혹은 삭제까지 하나의 명령어로 가능합니다.}
\item {서버를 유지할 시간을 명시해 서버 비용을 절감할 수 있습니다.}
\item {Pulumi와 Go를 사용했습니다.}
\end{cvitems}

\cventry
{Cloud-based malicious URL detection system.}
{gimi}
{\href{https://github.com/hanchchch/gimi}{Github}}
{09/2022 - 12/2022}
\begin{cvitems}
\item {URL을 격리된 컨테이너에서 검사하고, 악성 여부를 판단하는 시스템입니다.}
\item {FaaS처럼, URL 검사 요청이 들어왔을 때 새로운 컨테이너를 즉시 프로비저닝하고 검사합니다.}
\item {Heterogenous한 함수들을 구성하기 위해 간단한 FaaS 플랫폼을 직접 구현했습니다.}
\item {컨트롤러 및 마이크로서비스는 NestJS로 개발했으며, 메세지 큐는 Redis를, 컨테이너 관리는 Docker Go client로, 컴포넌트 간 통신는 protobuf와 gRPC를 사용했습니다.}
\end{cvitems}

\cventry
{Simulation for CSMA/CA with BEB and RTS/CTS}
{csma-ca}
{\href{https://github.com/hanchchch/csma-ca}{Github}}
{12/2021 - 12/2021}
\begin{cvitems}
\item {CSMA/CA 프로토콜의 시뮬레이션 환경을 파이썬으로 구현했습니다.}
\item {DI 패턴을 차용하여 RTS, frame rate, detect range 등 설정을 자유롭게 할 수 있습니다.}
\item {시뮬레이션은 터미널에 시각화되어 직관적으로 확인할 수 있으며 결과는 csv 파일로 저장됩니다.}
\item {많은 시뮬레이션 결과를 빠르게 얻기 위해 시각화를 해제하고 병렬로 진행시킬 수 있습니다.}
\item {실험 및 분석을 진행하여 해당 레포 README에 보고서로 작성하였습니다.}
\item {무선이동통신네트워크 과목의 final 과제로 진행한 프로젝트이며, 해당 과목은 A+로 이수하였습니다.}
\end{cvitems}
