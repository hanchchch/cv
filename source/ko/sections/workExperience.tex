\section{Work Experience}

\cventry
{MLOps Engineer} % Job Title
{Corca, Inc.} % Organization
{} % Location
{07/2021 -} % Date
\begin{cvitems}
\item {MLOps 관련 업무를 총괄했습니다.}
\item {데이터 파이프라인, CI/CD 및 CT 파이프라인을 구축했으며, 전반적인 시스템 설계 및 구현을 담당했습니다.}
\item {Go, gRPC와 protobuf를 사용해 고성능의 언어중립적 ML 모델 서버를 개발했습니다.}
\item {Kafka streams로 실시간 데이터 스트림을 구축했으며, 분당 \textasciitilde 30만 개의 메세지를 처리했습니다.}
\item {Redis를 채널로 하여 NestJS 서버와 Go 클라이언트가 참여하는 이벤트 드리븐 아키텍쳐를 구현했습니다.}
\item {Pulumi로 모든 인프라를 코드화하여 관리했습니다.}
\item {분당 \textasciitilde 30만 개의 요청을 성공적으로 처리했으며, 모든 컴포넌트를 scale-out 가능하도록 설계해 robust한 시스템을 구현했습니다.}
\end{cvitems}

\cventry
{Backend Developer} % Job Title
{Pangea, Inc.} % Organization
{} % Location
{10/2021 - 12/2021} % Date
\begin{cvitems}
\item {"Moment of Artist" 라는 네이버 LINE 블록체인을 기반으로 하는 K-pop 아이돌 컨텐츠 NFT 마켓플레이스를 개발하고 배포했습니다. (https://momentofartist.com)}
\item {NestJS로 RESTful API를 구성했습니다.}
\item {React, Next.js, redux-saga로 프론트엔드를 개발했습니다.}
\item {DevOps 관련 업무를 담당해, 1.5만 명의 사용자에게 안정적으로 서비스를 제공했습니다.}
\item {AWS ECS, AWS ElastiCache, MongoDB Atlas, CloudFlare 등 클라우드 기반 서비스를 활용했습니다.}
\end{cvitems}

\cventry
{Blockchain Developer} % Job Title
{Mars Green} % Organization
{} % Location
{07/2021 - 09/2021} % Date
\begin{cvitems}
\item {"Mars Green" 이라는 이더리움 기반 NFT 갤러리의 스마트 컨트랙트와 웹 백엔드를 설계 및 구현했습니다. (https://marsgreen.co)}
\item {Solidity 와 express.js, ethers.js 를 이용했습니다.}
\item {NFT 거래의 지속적인 트래킹과 웹사이트를 통한 컨트랙 관리 등을 가능하게 하기 위해 웹 백엔드와 스마트 컨트랙트를 연결하는 작업을 수행했습니다.}
\item {AWS ECS Fargate로 서비스를 배포하고 관리했습니다.}
\item {AWS CodePipeline로 CD 파이프라인을 구성했습니다.}
\end{cvitems}

\cventry
{Web Full Stack Developer} % Job Title
{CodeWings, Inc.} % Organization
{} % Location
{12/2020 - 07/2021} % Date
\begin{cvitems}
\item {"헬로알고" 라는 알고리즘 교육 플랫폼을 개발하고 관리했습니다.}
\item {Django RESTful Framework를 이용해 REST API를 구현했습니다.}
\item {gRPC와 socket.io로 마이크로서비스 간 연결을 구현했습니다.}
\item {기존 Django 만으로 구성된 웹서비스를 Next.js와 redux를 이용해 SPA 구조로 프론트엔드를 리팩토링했습니다.}
\item {AWS ECS Fargate와 EC2로 서비스를 배포했으며 CodePipeline으로 CD 파이프라인을 구성했습니다.}
\end{cvitems}
